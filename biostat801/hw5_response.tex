\documentclass{article}

\usepackage{amsmath, amssymb}
\newcommand{\B}{\mathcal{B}}

\title{Biostat 801 Hw 5}
\author{David Zhang}
\date{Oct 20, 2017}

\begin{document}

\maketitle

\begin{enumerate}
\item Problem 1 \\
    Proof: 
    \begin{enumerate}
    \item ($\Rightarrow$)
    Let $A \in \mathcal{F}$. 
    Then $A \perp A^C$,
    so $P(A) P(A^C) = P(AA^C) = P(\phi) = 0$.
    Thus $P(A) = 0 \;\text{or}\; 1$. \\
    \item ($\Leftarrow$):
    Let $A, B \in \mathcal{F}$.
    Then $P(AB) = 0 \;\text{or}\; 1$.
    \begin{enumerate}
    \item If $P(AB) = 0$,
    then $P(A) = 0$ or $P(B) = 0$, so $P(A) P(B) = 0$.
    \item If $P(AB) = 1$,
    then $P(A) = 1$ and $P(B) = 1$, so $P(A) P(B) = 1$.
    \end{enumerate}
    In both cases,
    $P(AB) = P(A)P(B)$,
    so $A \perp B$.
    \end{enumerate}
\item Problem 2 \\
    Let $(\Omega, \B, P) = ([0,1], \B([0,1]), \lambda)$,
    where $\lambda$ is the Lebesgue measure.
    Define
    \[
    X(\omega) := 
    \begin{cases}
    -1, & \text{if } \omega \in [0, \frac{1}{2}] \\
    1, & \text{if } \omega \in (\frac{1}{2}, 1]
    \end{cases}
    \]
    and $Y := -X$.
    Then clearly $Y \not\perp X$.
    However, $Y^2 = X^2 = 1$,
    so $Y^2 \perp X^2$ (see Problem 1).
\item Problem 3 \\
Proof:
\begin{enumerate}
\item $(\Leftarrow)$:
    Suppose $P[X = a] = 1$.
    Let $A, B \in \B$.
    \begin{enumerate}
    \item If $a \in AB$,
        then $P[X \in AB] = 1$,
        and at the same time $P[X \in A] = P[X \in B] = 1$,
        since $a \in A$ and $a \in B$.
    \item If $a \notin AB$,
        then $P[X \in AB] = 0$,
        and at the same time $P[X \in A] = 0$ or $P[X \in B] = 0$,
        since $a \notin A$ or $a \notin B$.
    \end{enumerate}
    In both cases,
    $P[X \in AB] = P[X \in A] P[X \in B]$,
    so $X \perp X$.
\item $(\Rightarrow)$:
    Suppose $X \perp X$.
    Let $A \in \B$.
    Then
\[
0 = P(\phi) = P(AA^C) = P(A) P(A^C),
\]
so $P(A) = 0 \text{ or } 1$ for all $A \in \B$.
Let $F(x) = P[X \leq x]$.
Then $F(x) = 0 \text{ or } 1$ for all $x$.
Define $S := \{x: F(x) = 1\}$ and $b := \inf S$.
Since $F(\infty) = 1$ and $F(-\infty) = 0$,
$b \neq -\infty \text{ or } \infty$.
Moreover, 
$F$ is non-decreasing and right-continuous,
so $b \in S$
and thus $P[X \leq b] = 1$.
Further more,
\[
P[X < b] = P(\bigcup_{n=1}^\infty [X \leq b - \frac{1}{n}]) = \lim_{n \to \infty} P[X \leq b - \frac{1}{n}] = 0, 
\]
so
\[
P[X = b] = P[X \leq b] - P[X < b] = 1 - 0 = 1
\]
\end{enumerate}
\item Problem 4 \\
To show:
\[
E[\cdots E[X | \B_1] \cdots | \B_n] = E[X | \B_n] \quad a.s.
\]
Proof: \\
First consider $E[E[X|\B_1] | \B_2]$.
Let $B_2 \in \B_2$.
Then 
\[
\int_{B_2} E[ E[X|\B_1] | \B_2] dP 
= \int_{B_2} E[X|\B_1] dP 
= \int_{B_2} X dP 
= \int_{B_2} E[X|\B_2] dP,
\]
since $\B_2 \subset \B_1$.
By the Integral Comparison Lemma,
\[
E[ E[X | \B_1] | B_2] = E[X | \B_2] \quad a.s.
\]
Then by mathematical induction,
\[
E[\cdots E[X | \B_1] \cdots | \B_n] = E[X | \B_n] \quad a.s.
\]
\end{enumerate}

\end{document}
