\documentclass{article}

\usepackage{amsmath}

\title{Biostat 653 Homework 5}
\author{David (Daiwei) Zhang}

\begin{document}
\maketitle

\section{Solutions}

\begin{enumerate}
\item Problem 1
  \begin{enumerate}
  \item The GEE is
    \[
      \sum_{i=1}^N D_i^T V_i^{-1} (Y_i - \mu_i) = 0
    \]
    Here $D_i = \frac{\partial \mu}{\partial\mu} = I_2$, $V_i = I_2$,
    and $\mu_i = \mu$, so
    \[
      \hat{\mu} = \frac{1}{N} \sum^N Y_i
    \]
    and
    \[
      \hat{\delta} = L \hat{\mu} = \frac{1}{N} \sum^N (Y_{i2} -
      Y_{i1})
    \]
    where $L = [-1, 1]$.
  \item We have the empirical variance estimator
    \[
      Cov(\hat{\beta}) = F^{-1} G F^{-1}
    \]
    where
    \begin{align*}
      F = & \sum^N D_i^T V_i^{-1} D_i = NI_2\\
      G = & \sum^N D_i^T V_i^{-1} (\hat{Y}_i - \mu_i)(\hat{Y}_i - \mu_i)^T V_i^{-1} D_i
            = \sum^N (Y_i - \mu)(Y_i - \mu)^T
    \end{align*}
    Then
    \[
      \hat{Cov}(\hat{\mu}) = \frac{1}{N^2} \sum^N (Y_i - \mu)(Y_i -
      \mu)^T
    \]
    and
    \[
      \hat{Cov}(\hat{\delta}) = \frac{1}{N^2} L [\sum^N (Y_i -
      \mu)(Y_i - \mu)^T] L^T
    \]
    
  \item The variable
    \[
      \gamma = \beta_2 - \beta_1 = \log(\frac{\mu_2}{\mu_1})
    \]
    is the log ratio of the expected number of new cancers.
  \item For GLM, the usual variance function is $v(\mu) = \mu$, so in
    our case, we can use
    \begin{align*}
      V_i = & \phi A^{1/2} R(\alpha) A^{1/2} \\
      = & \phi
          \begin{bmatrix}
            \sqrt{\exp \beta_1} & 0 \\
            0 & \sqrt{\exp \beta_2}
          \end{bmatrix}
                \begin{bmatrix}
                  1 & \alpha \\
                  \alpha & 1
                \end{bmatrix}
                           \begin{bmatrix}
                             \sqrt{\exp \beta_1} & 0 \\
                             0 & \sqrt{\exp \beta_2}
                           \end{bmatrix}
      \\ = & \phi
             \begin{bmatrix}
               \exp \beta_1 & \alpha \exp\frac{1}{2}(\beta_1 + \beta_2)\\
               \alpha \exp\frac{1}{2}(\beta_1 + \beta_2) & \exp
               \beta_2
             \end{bmatrix} = V
    \end{align*}
  \item For the GEE, we now have
    \[
      D_i = \frac{\partial \mu}{\partial \beta} =
      \begin{bmatrix}
        \exp \beta_1 & 0 \\
        0 & \exp \beta_2
      \end{bmatrix}
      = D
    \]
    so we need to solve
    \[
      0 = \sum^N D_i W (Y_i - \mu) = \sum^N D W (Y_i - \mu) = D W
      \sum^N (Y_i - \mu).
    \]
    A solution of this equation is $\hat{\mu} = \frac{1}{N} \sum^N Y_i$.
  \item Finally, for the variance estimate,
    \begin{align*}
      F = & NDV^{-1}D \\
      G = & D V^{-1} [\sum^N (Y_i - \mu) (Y_i - \mu)^T] V^{-1} D
    \end{align*}
    so
    \begin{align*}
      \hat{Cov}(\hat{\beta}) = & F^{-1} G F^{-1} \\
      = & \frac{1}{N^2} (D^{-1} V D^{-1}) (D V^{-1} [\sum^N (Y_i - \mu) (Y_i - \mu)^T] V^{-1} D) (D^{-1} V D^{-1}) \\
      = & \frac{1}{N^2} D(\hat{\beta})^{-1} [\sum^N (Y_i - \exp\hat{\beta}) (Y_i - \exp\hat{\beta})^T] D(\hat{\beta})^{-1}
    \end{align*}
  \end{enumerate}
\item Problem 2
  \begin{enumerate}
  \item See SAS output (proc genmod). The estimated treatment effect
    is $-0.0777$ with a p-value of $0.1485$.
  \item Increase in the expected log odds of the probability of having
    moderate or severe symptom per month in the control group.
  \item Difference in the expected rate of increase in log odds of the
    probability of having moderate or severe symptom per month between
    the case and the control group.
  \item Since the p-value for the intersection term is
    $0.1485 > 0.05$, the effect of the treatment is not significance.
  \item See SAS output (proc glimmix). The estimated treatment effect
    is $-0.1424$ with a p-value of $0.0284$.
  \item We have $\sigma^2_b = 16.0349$.  The magnitude of this value
    represents the variability of the log odds among all the
    individuals.
  \item Difference in the expected rate of increase in log odds of the
    probability of having moderate or severe symptom per month between
    the case and the control group, given that the individual's random
    intercept is fixed.
  \item Since the p-value for the intersection term is
    $0.1485 > 0.05$, the effect of the treatment is not significance,
    given that the individual's random intercept is fixed.
  \item The estimated treatment effect in the GLMM model, compared to
    that in the GEE model, is greater in magnitude and has a lower
    p-value. The GEE model treats all the inter-individual variation
    as measurement error, which adds noise to the analysis and reduces
    the effect and significance of the treatment. This issue is fixed
    in the GLMM model, which takes into account the inter-individual
    variation.
  \item We find that the number of quadrature points has virtually no
    effect on the estimation and p-value of the treatment effect.  See
    the table below. (SAS output truncated due to the excessive number of pages.)
    \begin{center}
      \begin{tabular}{|c|c|c|}
        \hline
        Quad Pts & Est Effect & p-val \\
        \hline
        2 & -0.1303 & 0.0271 \\
        5 & -0.1387	& 0.0276 \\
        10 & -0.1432	& 0.0286 \\
        20 & -0.1424	& 0.0284 \\
        30 & -0.1424	& 0.0284 \\
        50 & -0.1303	& 0.0271 \\
        \hline
      \end{tabular}
    \end{center}
  \end{enumerate}
\end{enumerate}

\section{SAS code}


\begin{verbatim}
libname bs653 "~/biostat653";
data toe;
	set bs653.toenail;
run;

proc genmod data=toe descending;
	class id;
	model y = month trt*month/d=bin;
	repeated subject=id/type=exch;
run;

PROC GLIMMIX METHOD=QUAD(QPOINTS=50);
	title "QP = 50";
	CLASS id;
	MODEL y= month trt*month /DIST=BINOMIAL LINK=LOGIT S;
	RANDOM INTERCEPT / SUBJECT=id TYPE=UN;
run;

PROC GLIMMIX METHOD=QUAD(QPOINTS=2);
	title "QP = 2";
	CLASS id;
	MODEL y= month trt*month /DIST=BINOMIAL LINK=LOGIT S;
	RANDOM INTERCEPT / SUBJECT=id TYPE=UN;
run;

PROC GLIMMIX METHOD=QUAD(QPOINTS=5);
	title "QP = 5";
	CLASS id;
	MODEL y= month trt*month /DIST=BINOMIAL LINK=LOGIT S;
	RANDOM INTERCEPT / SUBJECT=id TYPE=UN;
run;

PROC GLIMMIX METHOD=QUAD(QPOINTS=10);
	title "QP = 10";
	CLASS id;
	MODEL y= month trt*month /DIST=BINOMIAL LINK=LOGIT S;
	RANDOM INTERCEPT / SUBJECT=id TYPE=UN;
run;

PROC GLIMMIX METHOD=QUAD(QPOINTS=20);
	title "QP = 20";
	CLASS id;
	MODEL y= month trt*month /DIST=BINOMIAL LINK=LOGIT S;
	RANDOM INTERCEPT / SUBJECT=id TYPE=UN;
run;

PROC GLIMMIX METHOD=QUAD(QPOINTS=30);
	title "QP = 30";
	CLASS id;
	MODEL y= month trt*month /DIST=BINOMIAL LINK=LOGIT S;
	RANDOM INTERCEPT / SUBJECT=id TYPE=UN;
run;
\end{verbatim}


\end{document}
