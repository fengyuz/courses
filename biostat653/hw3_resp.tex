\documentclass{article}[12pt]
\usepackage{listings}

\title{Biostat 653 Homework 3}
\author{David (Daiwei) Zhang}

\begin{document}

\maketitle

\section{6.1.1}

See the SAS output.
Weight increases over time in all the three groups.
Group 1 and 3 grows approximately linearly and at the same rate.
Group 2 grows linearly before week 2, and after that it still grows linearly but the rate is reduced.

\section{6.1.2}

See the SAS code.

\section{6.1.3}

Since the observations are balanced and the rate of increase is assumed to be constant,
we treat time as a continuous variable.
See the SAS output.
Since the p-value for the Test 3 test of time $\times$ group is less thant $0.0001$, we reject the null hypothesis at $\alpha = 0.05$ and conclude that the rate of increase is significantly different between at least two of the three groups.

\section{6.1.4}

See the SAS output.

\section{6.1.5}
Expected increase in mean weight:
\begin{enumerate}
\item Group 1: $26.2151$
\item Group 2: $26.2151 - 7.0963 = 19.1188$
\item Group 3: $26.2151 -2.0944 = 24.1207$
\end{enumerate}

\section{6.1.6}

Group 2 shows a clear ``bend'' at time=2, so we add a linear spline after that.
To test whether the effect of the spline is significant,
we use a contrast to test whether the coefficients for $time_2$, $time_2*I(Group=2)$, and $time_2*I(Group=3)$ are all equal to zero.
See the SAS output.
Since the p-value for the contrast is $0.0003$,
we reject the null hypothesis at $\alpha=0.05$ and conclude that the effect of the spline is significant.

\section{6.1.7}
The BIC for the linear spline model ($884.8$) is lower than the BIC of the linear model ($891.8$).
Moreover, our LRT statistic is
\[
  l = (829.2 - 812.3) = 16.9  > 7.8147 = \chi^2_{3, 0.95},
\]
so we should use the spline model to represent hte pattern of change.

\section{6.1.8}

Additive 2 significantly reduces the rate of increase for weight,
and the effect is even stronger after the second week.

Additive 3 also significantly reduces the rate of increase for weight,
but the effect after week 2 is not significantly different from before week 2.

\pagebreak

\section{SAS code}

\begin{lstlisting}
libname bs653 "~/biostat653";

data ratuniv;
set bs653.rat;
time=0; wt=Y1; output;
time=1; wt=Y2; output;
time=2; wt=Y3; output;
time=3; wt=Y4; output;
time=4; wt=Y5; output;
drop Y1 Y2 Y3 Y4 Y5;
run;

proc sort data = ratuniv;
	by group time;
run;

proc means data = ratuniv noprint;
	var wt;
	by group time;
	output out = ratmean 
		Mean(wt) = meanWt;
run;

proc sgplot data = ratmean;
	series x = time y = meanWt / markers group = group;
run;

proc mixed data=ratuniv method=ml;
class ID Group(ref="1");
model wt=time Group*time/solution chisq outp=ratuniv_pred;
repeated/type=un subject=ID;
run;

proc sgplot data = ratuniv_pred;
	series x = time y = pred / markers group = group;
run;

data ratuniv_spline;
set ratuniv;
time_2 = max(time-2, 0);
run;

proc mixed data=ratuniv_spline method=ml;
class ID Group(ref="1");
model wt=time time_2 Group*time Group*time_2/solution chisq outp=ratuniv_spline_pred;
repeated/type=un subject=ID;
contrast "time_2 and Group*time_2" time_2 1, Group*time_2 1 -1 0, Group*time_2 1 0 -1/e;
run;

proc sgplot data = ratuniv_spline_pred;
	series x = time y = pred / markers group = group;
run;
\end{lstlisting}
\end{document}
