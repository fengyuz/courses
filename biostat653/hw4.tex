\documentclass{article}

\title{Biostat 653 Homework 4}
\author{David (Daiwei) Zhang}

\begin{document}
\maketitle

\section{Solutions}

\begin{enumerate}
\item See SAS output.
\item See SAS code.
\item Model fitting
  \begin{enumerate}
  \item 9.9530
  \item 0.03433
  \item -0.02883
  \item Approximately 95\% of the subjects in Program 2 have baseline measures of strength between $80.1324 \pm 1.96 \sqrt{9.9530} = (73.94892, 86.31588)$ \\
  Approximately 95\% of the subjects in Program 1 have baseline measures of strength between $81.2638 \pm 1.96 \sqrt{9.9530} = (75.07952, 87.44648)$.
  \item Approximately 95\% of the subjects in Program 1 have a rate of increase for their strenth with respect to time between $0.11702 \pm 1.96 \sqrt{0.03433} = (-0.2461358, 0.4801758)$ \\
  Approximately 95\% of the subjects in Program 2 have a rate of increase for their strenth with respect to time between $0.1690 \pm 1.96 \sqrt{0.03433} = (-0.1941558, 0.5321558)$
  \end{enumerate}
\item We have $LRS = 881.2 - 818.5 = 62.7 > 5.14$,
  so should include the random slope.
  The AIC and BIC also support this conclusion.
\item Effects:
  \begin{enumerate}
  \item Program 1, intercept: $81.2638 - 1.1314 = 80.1324$
  \item Program 1, slope: $0.1690 - 0.05198 = 0.11702$
  \item Program 2, intercept: $81.2638$
  \item Program 2, slope: $0.1690$
  \end{enumerate}
  \item The rate of increase in strength in Program 1 is expected to be $0.05198$ lower than that in Program 2. However, this effect is not significant since $0.4420 > 0.05$.
  \item We have $\hat{V}(Y_{i1} | b_i) = 0.6647$ and $\hat{V}(Y_{i1}) = 10.6177$.
    The former is the within-subject variance,
    while the latter is the across-subject variance plus within-subject variance.
  \item See SAS output.
  \item OLS: \\
    Intercept: 87.80000 \\
    Slope: 0.45000
  \item Random intercept and slope: \\
    Intercept: $81.2638 + 6.7042 = 87.968$ \\
    Slope: $0.1690 + 0.2163 = 0.3853$ \\
    The estimation is slightly different in the two models.
    For the mixed model, the estimation of the random effects for each subject depend on the estimation of the fixed effects,
    which in turn depend on all the subjects' response.
    OLS only takes into account the response of one subject.
\end{enumerate}

\section{SAS code}


\begin{verbatim}
data exercise;
	infile "~/biostat653/exercise.txt";
	input id program y1 y2 y3 y4 y5 y6 y7;
run;

data exercise_uni;
	set exercise;
	time = 0; strength = y1; output;
	time = 2; strength = y2; output;
	time = 4; strength = y3; output;
	time = 6; strength = y4; output;
	time = 8; strength = y5; output;
	time = 10; strength = y6; output;
	time = 12; strength = y7; output;
	drop y1 y2 y3 y4 y5 y6 y7;
run;

proc sort data = exercise_uni;
	by program time;
run;

proc means data=exercise_uni noprint;
	var strength;
	by program time;
	output out=exercise_mean
		mean(strength) = strength_mean;
run;

proc sgplot data=exercise_mean;
	title "Mean strength over time by group";
  	styleattrs
		datacontrastcolors=(black) 
     	datalinepatterns=(dot solid);
	series x = time y = strength_mean / markers group = program;
run;

proc mixed data = exercise_uni;
	title "Random intercept only";
	class id program;
	model strength = program time time*program / solution;
	random intercept
		 / type=un subject=id g gcorr v vcorr solution;
run;

proc mixed data = exercise_uni;
	title "Random intercept and slope";
	class id program;
	model strength = program time time*program / solution;
	random intercept time
		 / type=un subject=id g gcorr v vcorr solution;
run;

data exercise_id24;
	set exercise_uni;
	if id = 24;
run;

proc reg data=exercise_id24;
	title "OLS for id=24";
	model strength = time;
run;
	
\end{verbatim}


\end{document}
