\documentclass{amsart}

\begin{document}


Gelman 250.
Point hypothesis testing $\tehta = \theta_0$ has no meaning in Bayesian data analysis.
Asking for the distribution of $\theta$ is more useful than asking if $\theta$ is zero.

Press Ch 9
Fisher started the idea of rejecting the null hypothesis.
Influence by Popper's philosophy of falsifiability for science.

Problems with frequentist approach:
\begin{enumerate}
\item Hypothesis testing is not frequently needed in Bayesian frame.
\item Frequentist cannot put a probability and therefore a degree of belief on hypothesis
\item Significance level is arbitrary. In comparison, in Bayesian all we need is to see which hypothesis is favored by the evidence.
\item We can reject (or not reject) any hypothesis by making $n$ large (or small) enough.
\item Frequentist HT: sample space is divided into two regions, but these two regions contain observable values that have never been observed. This violates the likelihood principle.
  \item Frequentist hypothesis is often more conservative
\end{enumerate}

Lindley's vague prior procedure for Bayesian HT: use vague prior and the result is the same as frequentist interval testing.
Lindley paradox: We can always put enough weight on $H_0$ to make it not rejectable.

Jeffrey's method: use posterior odd ratio (average if necessary)

Gelman 61 - 65: Noninformative prior

Option 1: constant over the whole real line. Improper, but posterior might be proper
Option 2: Jeffrey's invariance principle
Problems for Jeffrey:
Controversial for multiparameter models


Usually the likelihood will dominate the posterior.
Different noninformative priors are the same for location families and scale families

Challenges:
One should avoid trying to find ``the'' prior distribution
A flat density may no longer be flat after reparametrization.

Useful when:
quantitizing prior info is not worth the effort

Goldstein

Reason to use subjective bayes: system too complex for traditional methods,
like software testing, subjective bayes is more efficient, because it provides a way to quantify experts' experience in software testing

Subjectivism in scientific enquiry

Difficulties:
1. Scientific inquiry should not include subjective elements.
2. The pain to gain ratio of subjective elements is too high.

Subjective bayes provides a way to quantify beliefs so that two people can trace where they disagree on (example: lady tasting tea vs ESP)

Can be used for self-assessment, or settling public opinions (need a prior that both agree on)

Response to objections to subjective bayes:
What is an acceptable alternative?
Objective bayes still measure degree of belief and therefore is not objective
On the other hand, conclusions from objective bayes analysis may be too careful and precise that it is no longer answering the question we are interested in
Also, when the conclusion is sensitive to the choice of priors, the questions usually does not have universally accepted answers.
Labeling a statistical result as ``objective'' may be misleading to non-statisticians

Practical reasons for using subjective bayes
When the choice of prior does not influence the answer much
When we are making a preliminary study
When we have limited resource

Statisticians need to synthesize different elements of a study








\end{document}
