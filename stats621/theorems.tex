\documentclass{article}

\usepackage{amsmath, amssymb}
\usepackage{cleveref}

\newcommand{\mc}[1]{\mathcal{#1}}
\newcommand{\mb}[1]{\mathbb{#1}}
\newcommand{\as}{\;a.s.\;}
\newcommand{\pto}{\overset{P}{\to}}
\newcommand{\asto}{\overset{\as}{\to}}
\newcommand{\wto}{\overset{w}{\to}}
\newcommand{\dto}{\Rightarrow}
\newcommand{\lpto}{\overset{L_p}{\to}}
\newcommand{\iid}{\overset{\text{iid}}{\sim}}
\newcommand{\deq}{\overset{d}{=}}
\newcommand{\R}{\mathbb{R}}
\newcommand{\C}{\mathcal{C}}
\newcommand{\B}{\mathcal{B}}

\begin{document}

\section{Convergence}
\begin{enumerate}
\item Convergence a.s. implies convergence i.p.
\item
  \begin{enumerate}
  \item Convergence i.p. iff Cauchy i.p.
  \item $X_n \pto X$ iff each subsequence ${X_{n_k}}$ contains a
    further subsequence $X_{n_{k(i)}} \asto X$.
  \end{enumerate}
\item Let $EX_n < \infty$. Then ${X_n}$ is ui iff
  \begin{enumerate}
  \item $\sup_n \int_A |X_n| dP \to 0$ for $P(A) \to 0$ with
    $A \in \B$.
  \item $\sup_n E|X_n| < \infty$
  \end{enumerate}
\item $\{|X_n|^p\}$ is ui if $\sup_n E(|X_n|^{p+\delta}) < \infty$ for
  some $\delta > 0$.
\item Let $p \geq 1$ and $X_n \in L_p$. The following are equivalent:
  \begin{enumerate}
  \item $X_n$ is $L_p$ convergent.
  \item $X_n$ is $L_p$ Cauchy ($\|X_n - X_m\|_p \to 0$).
  \item $|X_n|^p$ is ui and $X_n$ converges ip.
  \end{enumerate}
\end{enumerate}

\section{Law of Large Numbers}
\begin{enumerate}
\item General weak law of large numbers: If $\{X_n\}$ are independent,
  \[
    \sum_{j=1}^n P[|X_j| > n] \to 0,
  \]
  and
  \[
    \frac{1}{n^2} \sum_{j=1}^n EX_j^21_{[|X_j| \leq n]} \to 0,
  \]
  then
  \[
    \frac{S_n - a_n}{n} \pto 0,
  \]
  where \[ a_n = \sum_{j=1}^n E(X_j 1_{[|X_j| \leq n]}).
  \]
\item If $\{X_n\}$ is independent, then TFAE:
  \begin{enumerate}
  \item $S_n$ converges i.p.
  \item $S_n$ is Cauchy i.p.
  \item $S_n$ converges a.s.
  \item $S_n$ is Cauchy a.s.
  \end{enumerate}
\item If $\{X_n\}$ is independent and $\sum_n Var(X_n) < \infty$, then
  $\sum_n(X_n - \mu_n)$ converges a.s. and $L_2$.
\item If $X_n$ is iid, then
  \begin{align*}
    E|X_1| & < \infty \Rightarrow \bar{X}_n \asto \mu \\
    EX_1^2 & < \infty \Rightarrow S_n^2 \asto \sigma^2
  \end{align*}
\item Let $X_n$ be independent. Then $\sum_n X_n$ converges a.s. if
  and only if there exists $c > 0$ such that
  \begin{enumerate}
  \item $\sum_n P[|X_n| >c] < \infty$
  \item $\sum_n Var(X_n 1_{[|X_n| \leq c]}) < \infty$
  \item $\sum_n E(X_n 1_{[|X_n| \leq c]})$ converges
  \end{enumerate}
\end{enumerate}

\section{Convergence in Distribution}
\begin{enumerate}
\item Thm: If $F$ is proper, then all four convergence are
  equivalent. (Write $F_n \Rightarrow F$.)
\item Thm (Scheffe):
  \[
    \sup_{B \in \B(\R)} |F_n(B) - F(B)| = \frac{1}{2} \int |f_n(x) -
    f(x)| dx
  \]
\item Thm (Baby Shorohod): If $X_n \Rightarrow X_0$, then there exist
  $X_n^\#$ on $([0,1], \B([0,1]), \lambda)$ such that
  $X_n \deq X_n^\#$ and $X_n^\# \asto X_0^\#$.
\item Thm (Continuous mapping): If $X_n \Rightarrow X_0$ and
  $h: \R \to \R$ satisfies $P[X_0 \in (\C(h))^c] = 0$, then
  $h(X_n) \Rightarrow h(X_0)$. Moreover, if $h$ is bounded, then
  $Eh(X_n)\to Eh(X_0)$.
\item Thm (Delta method):
  \[
    \sqrt{n} \left(\frac{g(\bar{X}) - g(\mu)}{\sigma g'(\mu)} \right)
    \Rightarrow N(0,1)
  \]
\item Thm (Portmanteau): Let $F_n$ be proper. TFAE:
  \begin{enumerate}
  \item $F_n \Rightarrow F_0$
  \item $Eg(X_n) \to Eg(X_0)$ for any bounded continuous
    $g: \R \to \R$.
  \item $F_n(A) \to F(A)$ for any $A \in \B(\R)$ with
    $F_0(\partial(A)) = 0$ (boundary has zero probability).
  \end{enumerate}
\item Thm (Slutsky): If $X_n \dto X$ and $X_n - Y_n \pto 0$, then
\item Thm (Convergence to type):
  \begin{enumerate}
  \item If
    \begin{align}\label{one}
      \frac{X_n - b_n}{a_n} \dto U, \quad \frac{X_n -
      \beta_n}{\alpha_n} \dto V,
    \end{align}
    then
    \begin{align}\label{two}
      \frac{\alpha_n}{a_n} \to A > 0, \quad \frac{\beta_n - b_n}{a_n}
      \to B \in \R
    \end{align}
    and
    \begin{align}\label{three}
      V \deq \frac{U-B}{A}.
    \end{align}
  \item If \cref{two} holds, then either of \cref{one} implies the
    other and \cref{three} holds.
  \end{enumerate}
\end{enumerate}

\section{Central Limit Theorem}
\end{document}
