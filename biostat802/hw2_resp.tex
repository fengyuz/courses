\documentclass[12pt]{article}
\usepackage{amsmath}
\usepackage{bm}
\title{Biostat 803 Homework 2}
\date{February 12, 2018}
\author{David (Daiwei) Zhang}
\begin{document}
\maketitle

\section{Exponential distribution with a change point}
\subsection{Posterior for $\xi$}
For a variable $\theta$, let $d\theta = \mu(d\theta)$
where $\mu$ is the Lebesgue measure.
We have
\begin{align*}
\pi(\boldsymbol{x} | \xi, \eta)
= & \prod_i^\xi \eta \exp\{ -\eta x_i \}
\prod_{\xi+1}^n c \eta \exp\{ -c \eta x_i \} \\
= & \eta^n c^{n - \xi} \exp\big\{ -\eta\Big[\sum_i^\xi x_i + c \sum_{\xi+1}^n x_i\Big]\Big\},
\end{align*}
so
\begin{align*}
  \pi(\xi, \eta | \bm{x})
  \propto & \pi(\bm{x} | \xi, \eta) \pi(\xi, \eta) \\
  = & \pi(\bm{x} | \xi, \eta) \pi(\xi) \pi(\eta) \\
  = & \eta^n c^{n - \xi} \exp\Big\{ -\eta\Big[\sum_i^\xi x_i + c \sum_{\xi+1}^n x_i\Big]\Big\} \pi(\xi) \pi(\eta),
\end{align*}
and
\begin{align*}
  \pi(\xi | \bm{x})
  = & \int \pi(\xi, \eta| \bm{x}) d\eta \\
  \propto & \int \eta^n c^{n - \xi} \exp\Big\{ -\eta\Big[\sum_i^\xi x_i + c \sum_{\xi+1}^n x_i\Big]\Big\} \pi(\xi) \pi(\eta) d\eta \\
  = & \frac{c^{n-\xi} n! \pi(\xi)}{\Big[\sum_1^\xi x_i + c\sum_{\xi+1}^n x_i\Big]^{n+1}}  \\
  \propto & \frac{c^{n-\xi} \pi(\xi)}{\Big[\sum_1^\xi x_i + c\sum_{\xi+1}^n x_i\Big]^{n+1}} 
\end{align*}

\begin{align*}
\pi(\boldsymbol{x} | \xi)
= & \int \pi(\boldsymbol{x} | \xi, \eta) \pi(\eta) d\eta \\
\propto & \int \pi(\boldsymbol{x} | \xi, \eta) d\eta \\
= & \frac{c^{n-\xi} n!}{[\sum_1^\xi x_i + c\sum_{\xi+1}^n x_i]^{n+1}} \\
= & \frac{n! c^{n-\xi}}{x_1^{n+1}} [\sum_1^\xi z_i + c\sum_{\xi+1}^n z_i]^{-n-1} \\
\propto & \frac{c^{n-\xi}}{x_1^{n+1}} [\sum_1^\xi z_i + c\sum_{\xi+1}^n z_i]^{-n-1} \\
\end{align*}
and thus
\begin{align*}
\pi(\xi | \boldsymbol{x})
  \propto & \pi(\boldsymbol{x} | \xi) \pi(\xi) \\
\propto & \frac{\pi(\xi)}{x_1^{n+1}} c^{n-\xi}[\sum_1^\xi z_i + c\sum_{\xi+1}^n z_i]^{-n-1} \\
\propto & \pi(\xi) c^{n-\xi}[\sum_1^\xi z_i + c\sum_{\xi+1}^n z_i]^{-n-1} 
\end{align*}

\subsection{$\bm{Z}$ is ancillary with respect to $\eta$}
We know that the exponential distribution family
is a location family with respect to $\eta$,
so for $1 \leq i \leq \xi$,
the distribution of $W_i = \eta X_i$ does not depend on $\eta$.
Then $Z_i = W_i / W_1$ does not depend on $\eta$
for $2 \leq i \leq \xi$.
Similarly, for $\xi+1 \leq i \leq n$,
the distribution of $W_i = c \eta X_i$ does not depend on $\eta$.
Then $Z_i = W_i / (cW_1)$ does not depend on $\eta$
for $\xi+1 \leq i \leq n$.
Thus $\bm{Z} = (Z_2, \ldots, Z_n)$ is ancillary wrt $\eta$.

\subsection{Likelihood for $\xi$ under $Z$}

We transform $(x_1, \ldots, x_n)$ to $(x_1, z_2, \ldots, z_n)$.
We have
\begin{align*}
  \pi(x_1, z_2, \ldots, z_n | \xi, \eta)
  = & \pi(x_1, z_2, \ldots, z_n | \xi, \eta) |\bm{J}| \\
  = & \eta^n c^{n - \xi} \exp\big\{ -\eta\Big[\sum_i^\xi x_i + c \sum_{\xi+1}^n x_i\Big]\Big\} x^{n-1} \\
  = & \eta^n c^{n - \xi} \exp\big\{ -\eta x_1 \Big[\sum_i^\xi z_i + c \sum_{\xi+1}^n z_i\Big]\Big\} x^{n-1},
\end{align*}
since
\begin{align*}
  \bm{J}=
  \begin{bmatrix}
    \frac{\partial x_1}{\partial x_1} &  \frac{\partial x_1}{\partial z_2} & \cdots & \frac{\partial x_1}{\partial z_n} \\ 
    \frac{\partial x_2}{\partial x_1} &  \frac{\partial x_2}{\partial z_2} & \cdots & \frac{\partial x_2}{\partial z_n} \\ 
    \vdots & \vdots & \ddots & \vdots \\
    \frac{\partial x_n}{\partial x_1} &  \frac{\partial x_n}{\partial z_2} & \cdots & \frac{\partial x_n}{\partial z_n} 
  \end{bmatrix}
  =
  \begin{bmatrix}
    1 & 0 & 0 & \cdots & 0 \\
    z_2 & x_1 & 0 & \cdots & 0 \\
    z_3 & 0 & x_1 & \cdots & 0 \\
    \vdots & \vdots & \vdots & \ddots & \vdots \\
    z_n & 0 & 0 & \cdots & 0
  \end{bmatrix},
\end{align*}
so $|\bm{J}| = x_1^{n-1}$. 
Then 
\begin{align*}
  \pi(z_2, \ldots, z_n | \xi, \eta)
  = & \int \pi(x_1, z_2, \ldots, z_n | \xi, \eta) d x_1 \\
  = & \int \eta^n c^{n - \xi} \exp\big\{ -\eta x_1 \Big[\sum_i^\xi z_i + c \sum_{\xi+1}^n z_i\Big]\Big\} x^{n-1} d x_1  \\
  = & \eta^n c^{n-\xi} n! \eta^{-n} \Big[\sum_i^\xi z_i + c \sum_{\xi+1}^n z_i\Big]^{-n} \\
  \propto & c^{n-\xi} \Big[\sum_i^\xi z_i + c \sum_{\xi+1}^n z_i\Big]^{-n},
\end{align*}
so $\pi(\bm{z} | \eta) = \pi(z_2, \ldots, z_n | \eta) = \pi(z_2, \ldots, z_n | \xi, \eta)$.
Notice that $\pi(z_2, \ldots, z_n | \xi, \eta)$ does not depend on $\eta$,
so this is another way to show that $(z_2, \ldots, z_n)$ is an ancillary statistic with respect to $\eta$.
Furthermore,
\begin{align*}
  \pi(\xi|\bm{z})
  \propto & \pi(\bm{z} | \eta) \pi(\xi) \\
  = & c^{n-\xi} \Big[\sum_i^\xi z_i + c \sum_{\xi+1}^n z_i\Big]^{-n} \pi(\xi)
\end{align*}

\subsection{Reconciliation of the two likelihoods}
Under $\pi(\eta) \propto 1$,
we have
$\pi(\xi|\bm{z}) / \pi(\xi | \bm{x}) = \sum_i^\xi z_i + c \sum_{\xi+1}^n z_i \neq 1$
in general,
so the two posteriors cannot reconcile.
However, if we use $\pi(\eta) \propto \eta^{-1}$,
then
\begin{align*}
  \pi(\xi|\bm{x})
  \propto & \int \eta^n c^{n - \xi} \exp\Big\{ -\eta\Big[\sum_i^\xi x_i + c \sum_{\xi+1}^n x_i\Big]\Big\} \pi(\xi) \pi(\eta) d\eta \\
  = & \int \eta^n c^{n - \xi} \exp\Big\{ -\eta\Big[\sum_i^\xi x_i + c \sum_{\xi+1}^n x_i\Big]\Big\} \eta^{-1} \pi(\xi) d\eta \\
  = & \int \eta^{n-1} c^{n - \xi} \exp\Big\{ -\eta\Big[\sum_i^\xi x_i + c \sum_{\xi+1}^n x_i\Big]\Big\} \pi(\xi) d\eta \\
  = & \frac{c^{n-\xi} n! \pi(\xi)}{\Big[\sum_1^\xi x_i + c\sum_{\xi+1}^n x_i\Big]^{n}}  \\ 
  = & \frac{c^{n-\xi} n! \pi(\xi)}{x_1^n \Big[\sum_1^\xi z_i + c\sum_{\xi+1}^n z_i\Big]^{n}} \\
  \propto & \frac{c^{n-\xi} \pi(\xi)}{\Big[\sum_1^\xi z_i + c\sum_{\xi+1}^n z_i\Big]^{n}} \\
  = & \pi(\xi|\bm{z}).
\end{align*}

\section{Uniform distribution with Pareto prior}

\subsection{Posterior}

Let $X_{(n)} = \max(X_i)$.
For the likelihood, we have
\begin{align*}
  \pi(\bm{x} | \theta) = \theta^{-n} I[X_{(n)} \leq \theta],
\end{align*}
so the posterior is
\begin{align*}
  \pi(\theta | \bm{x})
  \propto & \pi(\bm(x) | \theta) \pi(\theta) \\ 
  \propto & \theta^{-n} I[X_{(n)} \leq \theta] \alpha \beta^\alpha \theta^{-\alpha-1} I[\beta \leq \theta]\\
  = & \alpha \beta^\alpha \theta^{-a-n-1} I[\theta \geq \tilde{\beta}]
\end{align*}
where $\tilde{\beta} = \max(X_{(n)}, \beta)$.
To find the normalizing constant, we have
\begin{align*}
  \pi(\bm{x}) = &  \int \pi(\bm{x} | \theta) \pi(\theta) d\theta \\
  = & \int \alpha \beta^\alpha \theta^{-a-n-1} I[\theta \geq \tilde{\beta}] d\theta \\
  = & \alpha \beta^\alpha \int_{\tilde{\beta}}^\infty \theta^{-a-n-1} d\theta \\
  = & \alpha \beta^\alpha (\alpha+n)^{-1} \tilde{\beta}^{-\alpha-n}.
\end{align*}
Then
\begin{align*}
  \pi(\theta | \bm{x})
  = & \frac{\pi(\bm{x}|\theta) \pi(\theta)}{\pi(\bm{x})} \\
  = & (a+n) \tilde{\beta}^{\alpha+n} \theta^{-a-n-1} I[\theta \geq \tilde{\beta}] \\
  \sim & PA(\alpha+n, \max(X_{(n)}, \beta)).
\end{align*}

\subsection{Bayes estimator}

For the Bayes estimator under the square loss function,
we have the posterior mean
\begin{align*}
  \hat{\theta}_{Bayes}
  = & E[\theta | \bm{x}] \\
  = & \int \theta (a+n) \tilde{\beta}^{\alpha+n} \theta^{-a-n-1} I[\theta \geq \tilde{\beta}] d\theta \\
  = & (\alpha+n) \tilde{\beta}^{\alpha+n} \int_{\tilde{\beta}}^\infty \theta^{-a-n} d\theta \\
  = & (\alpha+n) \tilde{\beta}^{\alpha+n} (\alpha+n-1)^{-1} \tilde{\beta}^{-\alpha-n+1} \\
  = & \frac{\alpha+n}{\alpha+n-1} \max(X_{(n)}, \beta)
\end{align*}

\subsection{Compare Bayes with MLE}

Since $\beta < X_{(n)}$, we have
\[
  \hat{\theta}_{Bayes} = \frac{\alpha+n}{\alpha+n-1} X_{(n)} = (1+\epsilon) X_{(n)}.
\]
where $\epsilon = (\alpha + n -1)^{-1}$.
On the other hand, by looking at $\pi(\bm{x} | \theta)$,
we know
\[
  \hat{\theta}_{MLE} = X_{(n)}.
\]
We need to know the distribution of $X_{(n)}$.
For the CDF,
\[
  P(X_{(n)} \leq x) = \frac{x^n}{\theta^n},
\]
so the PDF is
\[
  \pi(X_{(n)}) = \theta^{-n} n x^{n-1}.
\]
Then
\begin{align*}
  E[X_{(n)}]
  = & \int_0^\theta x \theta^{-n} n x^{n-1} dx \\
  = & \theta^{-n} n \int_0^\theta x^{n} dx \\
  = & \theta^{-n} n (n+1)^{-1} \theta^{n+1}\\
  = & \frac{\theta n}{n+1}, \\
  E[X_{(n)}^2]
  = & \frac{\theta^2 n}{n+2}, \\
  Var[X_{(n)}]
  = & E[X_{(n)}^2] - E[X_{(n)}]^2 \\
  = & \frac{\theta^2 n}{(n+1)^2(n+2)}
\end{align*}
Hence
\begin{align*}
  MSE(\hat{\theta}_{MLE}) =
  & Bias[\hat{\theta}_{MLE}]^2 + Var[\hat{\theta}_{MLE}] \\
  = & \frac{\theta^2}{(n+1)^2} + \frac{\theta^2 n}{(n+1)^2(n+2)} \\
  = & \frac{2 \theta^2}{(n+1)(n+2)}.
\end{align*}
Similarly, for the Bayes estimator,
\begin{align*}
  Bias(\hat{\theta}_{Bayes})
  = & \frac{(\epsilon n - 1)\theta}{n+1} \\
  Var(\hat{\theta}_{Bayes})
  = & \frac{\theta^2 n (1+\epsilon)^2}{(n+1)^2(n+2)} \\
  MSE(\hat{\theta}_{Bayes})
  = & \frac{(\epsilon n - 1)^2(n+2) + n(1+\epsilon)^2}{(n+1)^2(n+2)} \theta^2.
\end{align*}
Then comparing the two estimators,
\begin{align*}
  MSE(\hat{\theta}_{Bayes}) - MSE(\hat{\theta}_{MLE}) 
  = & \frac{(\epsilon^2 n^2 - 2\epsilon n)(n+2) + n(\epsilon^2 + 2\epsilon)}{(n+1)^2(n+2)} \theta^2 \\
  = & [(\epsilon n - 2)(n+2) + (\epsilon + 2)] \frac{n \epsilon\theta^2}{(n+1)^2(n+2)} \\
  = & [\epsilon n^2 + 2\epsilon n + \epsilon -2n -2 ] \frac{n \epsilon\theta^2}{(n+1)^2(n+2)} \\
  = & [\epsilon (n+1)^2 -2(n+1)] \frac{n \epsilon\theta^2}{(n+1)^2(n+2)} \\
  = & [\epsilon (n+1) -2] \frac{(n+1)n \epsilon\theta^2}{(n+1)^2(n+2)} \\
  = & \frac{n+1-2\alpha-2n+2}{\alpha+n+1} \frac{(n+1)n \epsilon\theta^2}{(n+1)^2(n+2)} \\
  = & (3-2\alpha-n) \frac{(n+1)n \epsilon\theta^2}{(\alpha+n+1)(n+1)^2(n+2)} 
\end{align*}
Since the second term is always positive,
we conclude that the Bayes estimator has a smaller MSE than the MLE estimator
if and only if $2\alpha + n > 3$.

\section{Uniform distribution with a uniform prior}

\subsection{Bayes solution}

For the Bayes solution, we have
\begin{align*}
  \pi(\theta)
  = &  (2\alpha)^{-1} I[-\alpha \leq \theta \leq \alpha] \\
  \pi(\bm{x} | \theta)
  = & I[X_{(1)} \geq \theta - \frac{1}{2}]
      I[X_{(n)} \leq \theta + \frac{1}{2}] \\
  \pi(\theta | \bm{x} )
  \propto & (2\alpha)^{-1}
            I[\theta \geq \max(-\alpha, X_{(n)} - \frac{1}{2})]
            I[\theta \leq \min(\alpha, X_{(1)} + \frac{1}{2})] \\
  \propto & I[\theta \geq \max(-\alpha, X_{(n)} - \frac{1}{2})]
            I[\theta \leq \min(\alpha, X_{(1)} + \frac{1}{2})],
\end{align*}
so the Bayes estimator under the absolute error loss is the posterior mean,
that is,
\[
  \delta_{\Pi_\alpha}(\bm{x})
  = \frac{1}{2}
  [
  \max(-\alpha, X_{(n)} - \frac{1}{2}) +
  \min(\alpha, X_{(1)} + \frac{1}{2})
  ].
\]

\subsection{Limit of the Bayes estimator}

For every $\theta \in (-\infty, \infty)$,
we know $X_{(n)} \geq \theta - \frac{1}{2}$
and $X_{(1)} \leq \theta + \frac{1}{2}$.
Then for $\alpha$ large enough,
$-\alpha < \theta -1 \leq X_{(n)} - \frac{1}{2}$ and
$\alpha > \theta + 1 \geq X_{(1)} + \frac{1}{2}$.
Then
\[
  \delta_{\Pi_\alpha}(\bm{x})
  = \frac{1}{2}
  [
  X_{(n)} - \frac{1}{2}
  + X_{(1)} + \frac{1}{2}
  ]
  = \frac{1}{2}[ X_{(1)} + X_{(n)}]
  = \delta(\bm{x}).
\]
Thus $\delta_{\Pi_\alpha} \to \delta$ as $\alpha \to \infty$. 

\subsection{Minimax estimator}
It suffices to show that the limiting Bayes estimator has constant risk with respect to $\theta$.
The risk function is
\begin{align*}
  R(\delta, \theta)
  = & E[|\delta(\bm{x}) - \theta|] \\
  = & E\Big[\big|\frac{1}{2}(X_{(1)} + X_{(n)}) - \theta\big|\Big] \\
  = & \frac{1}{2}E\Big[\big|(X_{(1)} - \theta) + (X_{(n)} - \theta)\big|\Big] 
\end{align*}
However, $X_{(1)} - \theta$ and $X_{(n)} - \theta$ are ancillary statistics,
since the distribution family is a location family with respect to $\theta$.
Thus the risk function does not depend on $\theta$.
Thus $\delta$ is a minimax estimator.

\end{document}