\documentclass[12pt]{article}
\usepackage{amsmath}

\title{Biostat 802 Homework 3}
\author{David (Daiwei) Zhang}

\begin{document}

\maketitle

\section{Discrete Distributions}

We first find the distribution of the likelihood ratios
for $P_1$ and $P_2$ against $P_0$
under the measure of $P_0$.
We have
\begin{center}
  \begin{tabular}{|l|l|l l|l|l|l|l}
    \hline
    $L_{P_1}$ & 0.85 & 2.00 && 2.50 & 4.00 & $\infty$ \\
    $P_0(L_{P_1})$ & 0.92 & 0.04 && 0.02 & 0.02 & 0.00 \\
    \hline
    $x$ & 6 & 1 & 4 & 2 & 3 & 5 \\
    $P_0(x)$ & 0.92 & 0.03 & 0.01 & 0.02 & 0.02 & 0.00 \\
    \hline
  \end{tabular}
\end{center}
\begin{center}
  \begin{tabular}{|l|l|l|l|l|l|l|}
    \hline
    $L_{P_2}$ & 0.00 & 0.78 & 2.50 & 3.00 & 6.00 & $\infty$ \\
    $P_0(L_{P_2})$ & 0.01 & 0.92 & 0.02 & 0.03 & 0.02 & 0.00 \\
    \hline
    $x$ & 4 & 6 & 2 & 1 & 3 & 5 \\
    $P_0(x)$ & 0.01 & 0.92 & 0.02 & 0.03 & 0.02 & 0.00 \\
    \hline
  \end{tabular}
\end{center}
where $L_{p_1} = P_1(X) / P_0(X)$ and $L_{p_2} = P_2(X) / P_0(X)$.

Now in order to find a level-$\alpha$ test,
we pick the observations with the highest likelihood ratio
(the right most side of the two tables above).
\begin{enumerate}
\item $\alpha = 0.01$.
  For $P_1$,
  we should reject $X = 5$ and $\frac{1}{2}$ of $X = 3$,
  in order to achieve maximum power.
  By looking at the table for $P_2$,
  we find that power under $P_2$ is maximized by the same rejection rule.
  Thus the UMP test exists, which is
  \[
    \phi(x) = \left.
    \begin{cases}
      0, & \text{if  } x \in \{1, 2, 4, 6\}, \\
      \frac{1}{2}, & \text{if  } x = 3, \\
      1, & \text{if  } x = 5
    \end{cases}
    \right.
  \]
\item $\alpha = 0.05$.
  The situation is different here.
  For $P_1$,
  we must reject $X \in \{2,3,5\}$ and $\frac{1}{4}$ of $X \in \{1,4\}$
  to achieve maximum power.
  But for $P_2$,
  we must reject $X \in \{1,3, 5\}$ and nothing else,
  since $P_0[X \in \{1,3,5\}] = 0.05$ already.
  Thus a UMP test does not exist.
\item $\alpha = 0.07$.
  Now as we increase $\alpha$, the situation changes again.
  This time for $P_2$,
  we must reject $X \in \{2,1,3,5\}$ exactly to achieve maximum power.
  Luckily, this is allowed by the UMP rejection rule for $P_1$, too.
  For $P_1$,
  we must reject $X \in \{2, 3, 5\}$ with probability $1$ and $X \in \{1,4\}$ with probability $\frac{3}{4}$.
  Since $P_0[X=1] = 0.03$ and $P_0[X=4] = 0.01$,
  things work out exactly well if we reject $X=1$ but not $X=3$.
  Overall, the UMP test is
  \[
    \phi(x) = \left.
    \begin{cases}
      0, & \text{if  } x \in \{4, 6\}, \\
      1, & \text{if  } x \in \{1,2,3,5\}
    \end{cases}
    \right.
  \]
\end{enumerate}

      




\end{document}